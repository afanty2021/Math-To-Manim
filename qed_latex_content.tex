\documentclass[12pt]{article}

\usepackage{amsmath, amssymb, amsfonts}

\usepackage{geometry}

\usepackage[utf8]{inputenc}

\geometry{a4paper, margin=1in}

\usepackage{hyperref}

\hypersetup{

    colorlinks=true,

    linkcolor=blue,

    filecolor=magenta,      

    urlcolor=cyan,

}



\title{A Deeper Dive into the Mathematics of Quantum Electrodynamics}

\author{As Illustrated by the Manim Visualization}

\date{\today}



\begin{document}

\maketitle

\tableofcontents

\newpage



\section*{Introduction}

Quantum Electrodynamics (QED) is the relativistic quantum field theory of the electromagnetic interaction. It describes how light (photons) and matter (charged spin-1/2 particles, primarily electrons and positrons) interact. QED is renowned for its remarkable predictive accuracy and serves as the archetype for other gauge theories in the Standard Model of particle physics. The Manim visualization touched upon several key mathematical and physical concepts that form the bedrock of QED. This document elaborates on these concepts.



\section{Minkowski Spacetime and the Relativistic Metric}

The stage for relativistic physics, including QED, is Minkowski spacetime. This is a four-dimensional flat spacetime manifold where three dimensions are spatial and one is temporal.



\subsection{Four-Vectors}

Events in Minkowski spacetime are described by four-vectors $x^\mu = (x^0, x^1, x^2, x^3) = (ct, \mathbf{x})$, where $c$ is the speed of light, $t$ is time, and $\mathbf{x} = (x, y, z)$ represents the spatial coordinates. We use Greek indices ($\mu, \nu, \dots$) running from 0 to 3, and Roman indices ($i, j, \dots$) for spatial components (1 to 3). Einstein's summation convention (summation over repeated up-down indices) is implied.



\subsection{The Minkowski Metric}

The distance (or interval) between two infinitesimally separated events in Minkowski spacetime, $dx^\mu$, is given by the invariant line element $ds^2$:

\begin{equation}

ds^2 = \eta_{\mu\nu} dx^\mu dx^\nu = -(dx^0)^2 + (dx^1)^2 + (dx^2)^2 + (dx^3)^2

\end{equation}

In terms of $dt, dx, dy, dz$:

\begin{equation}

ds^2 = -c^2 dt^2 + dx^2 + dy^2 + dz^2

\end{equation}

Here, $\eta_{\mu\nu}$ is the Minkowski metric tensor, typically chosen with the signature (-,+,+,+):

\begin{equation}

\eta_{\mu\nu} = \begin{pmatrix} -1 & 0 & 0 & 0 \\ 0 & 1 & 0 & 0 \\ 0 & 0 & 1 & 0 \\ 0 & 0 & 0 & 1 \end{pmatrix}

\end{equation}

The invariance of $ds^2$ under Lorentz transformations is a cornerstone of Special Relativity.



\subsection{The Light Cone}

The light cone structure at each point in spacetime is defined by $ds^2 = 0$.

\begin{itemize}

    \item $ds^2 < 0$: Timelike interval. Events are causally connected; one can influence the other.

    \item $ds^2 > 0$: Spacelike interval. Events are causally disconnected.

    \item $ds^2 = 0$: Null or lightlike interval. Events are connected by light signals.

\end{itemize}

The visualization's rotating wireframe 4D Minkowski spacetime with a light cone illustrates this causal structure. The highlighting of the negative time component and positive spatial components in the metric equation emphasizes the distinct nature of time and space in this geometry.



\section{Classical Electromagnetism: Maxwell's Equations}

QED builds upon classical electrodynamics, described by Maxwell's equations.



\subsection{Vector Calculus Notation (in vacuum, no sources for simplicity initially)}

The undulating plane waves ($\mathbf{E}$ for electric field, $\mathbf{B}$ for magnetic field) oscillating perpendicularly and propagating are solutions to the source-free Maxwell's equations:

\begin{align}

\nabla \cdot \mathbf{E} &= 0 \quad \text{(Gauss's Law for electricity, source-free)} \\

\nabla \cdot \mathbf{B} &= 0 \quad \text{(Gauss's Law for magnetism)} \\

\nabla \times \mathbf{E} &= -\frac{\partial \mathbf{B}}{\partial t} \quad \text{(Faraday's Law of Induction)} \\

\nabla \times \mathbf{B} &= \mu_0 \epsilon_0 \frac{\partial \mathbf{E}}{\partial t} \quad \text{(Ampère-Maxwell Law, source-free)}

\end{align}

where $c^2 = 1/(\mu_0 \epsilon_0)$.



\subsection{Relativistic Covariant Formulation}

Maxwell's equations can be written more elegantly and manifestly Lorentz covariant using tensor notation.

\paragraph{The Four-Potential:}

The electric scalar potential $\phi$ and magnetic vector potential $\mathbf{A}$ are combined into a four-vector potential:

\begin{equation}

A^\mu = (\phi/c, \mathbf{A})

\end{equation}

\paragraph{The Electromagnetic Field Strength Tensor:}

The electric and magnetic fields are components of the antisymmetric electromagnetic field strength tensor $F^{\mu\nu}$:

\begin{equation}

F^{\mu\nu} = \partial^\mu A^\nu - \partial^\nu A^\mu

\end{equation}

where $\partial^\mu = \eta^{\mu\nu}\partial_\nu = (-\frac{1}{c}\frac{\partial}{\partial t}, \nabla)$.

The components are:

\begin{equation}

F^{\mu\nu} = \begin{pmatrix}

0 & -E_x/c & -E_y/c & -E_z/c \\

E_x/c & 0 & -B_z & B_y \\

E_y/c & B_z & 0 & -B_x \\

E_z/c & -B_y & B_x & 0

\end{pmatrix}

\end{equation}

\paragraph{Maxwell's Equations in Tensor Form:}

The two inhomogeneous Maxwell's equations (Gauss's law for $\mathbf{E}$ with sources $\rho$, and Ampère-Maxwell law with current density $\mathbf{J}$) are combined into one tensor equation:

\begin{equation}

\partial_\mu F^{\mu\nu} = \mu_0 J^\nu \label{eq:inhom_maxwell}

\end{equation}

where $J^\nu = (c\rho, \mathbf{J})$ is the four-current density. The animation showed this form, emphasizing the transition.



The two homogeneous Maxwell's equations (Gauss's law for $\mathbf{B}$ and Faraday's law) are combined into:

\begin{equation}

\partial_\lambda F_{\mu\nu} + \partial_\mu F_{\nu\lambda} + \partial_\nu F_{\lambda\mu} = 0

\end{equation}

This can also be written using the Levi-Civita tensor $\epsilon^{\alpha\beta\mu\nu}$ as $\epsilon^{\alpha\beta\mu\nu}\partial_\beta F_{\mu\nu} = 0$. These are automatically satisfied if $F_{\mu\nu}$ is derived from $A_\mu$ as $F_{\mu\nu} = \partial_\mu A_\nu - \partial_\nu A_\mu$.



The transformation from vector calculus to the compact tensor form $\partial_\mu F^{\mu\nu} = \mu_0 J^\nu$ elegantly underscores the relativistic nature of electromagnetism.



\section{The Lagrangian Density for Quantum Electrodynamics ($\mathcal{L}_{\text{QED}}$)}

The dynamics of a quantum field theory are encoded in its Lagrangian density $\mathcal{L}$. For QED, it combines the Dirac Lagrangian for fermions and the Maxwell Lagrangian for photons, plus an interaction term.

\begin{equation}

\mathcal{L}_{\text{QED}} = \mathcal{L}_{\text{Dirac}} + \mathcal{L}_{\text{EM}} + \mathcal{L}_{\text{int}}

\end{equation}

The explicit form shown in the animation is:

\begin{equation}

\mathcal{L}_{\text{QED}} = \bar{\psi}(i\gamma^\mu D_\mu - m)\psi - \frac{1}{4}F_{\mu\nu}F^{\mu\nu} \label{eq:qed_lagrangian}

\end{equation}

(Using natural units where $\hbar=c=1$. The elementary charge $e$ is absorbed into $D_\mu$).



\subsection{Components of the QED Lagrangian:}

\begin{itemize}

    \item $\psi(x)$: The Dirac spinor field, a four-component complex field operator that annihilates a fermion (e.g., an electron) or creates an anti-fermion (e.g., a positron) at spacetime point $x$. Highlighted in orange in the visualization.

    \item $\bar{\psi}(x) = \psi^\dagger(x)\gamma^0$: The Dirac adjoint spinor field operator.

    \item $\gamma^\mu$: The Dirac gamma matrices, a set of four $4 \times 4$ matrices satisfying the Clifford algebra:

    \begin{equation}

    \{\gamma^\mu, \gamma^\nu\} = \gamma^\mu\gamma^\nu + \gamma^\nu\gamma^\mu = 2\eta^{\mu\nu}I_4

    \end{equation}

    where $I_4$ is the $4 \times 4$ identity matrix. These are crucial for describing spin-1/2 particles in a relativistically covariant way. Highlighted in bright teal.

    \item $m$: The mass of the fermion (e.g., electron mass).

    \item $D_\mu$: The covariant derivative. This is key for introducing interactions in a gauge-invariant manner. It replaces the ordinary partial derivative $\partial_\mu$.

    \begin{equation}

    D_\mu = \partial_\mu + iqA_\mu(x)

    \end{equation}

    where $q$ is the electric charge of the fermion field $\psi$. For electrons, $q=-e$ (where $e$ is the positive elementary charge). The visualization used a convention often found in QFT texts where $e$ can be the coupling strength and may have its sign absorbed, or where $q$ itself represents the signed coupling. In the Manim code it was defined with a `+ie` or `+iq` to absorb the charge. Highlighted in green.

    \item $F_{\mu\nu} = \partial_\mu A_\nu - \partial_\nu A_\mu$: The electromagnetic field strength tensor, as defined before. $A_\mu(x)$ is the photon field operator. This term represents the kinetic energy and dynamics of the electromagnetic field (photons). Highlighted in gold.

    \item $\bar{\psi}(i\gamma^\mu \partial_\mu - m)\psi$: This would be the Lagrangian for a free Dirac field.

    \item $\bar{\psi}(i\gamma^\mu (iqA_\mu))\psi = -q\bar{\psi}\gamma^\mu\psi A_\mu$: This is the interaction term, $J^\mu A_\mu$, where $J^\mu = q\bar{\psi}\gamma^\mu\psi$ is the electromagnetic four-current of the Dirac field.

    \item $-\frac{1}{4}F_{\mu\nu}F^{\mu\nu}$: This is the Lagrangian for the free electromagnetic field. It is equivalent to $\frac{1}{2}(\mathbf{E}^2 - \mathbf{B}^2)$ in terms of field strengths when written in non-covariant form and including factors of $c, \epsilon_0$.

\end{itemize}

The pulsating symbols in the animation rightly suggest that $\psi$, $A_\mu$ (and thus $D_\mu$ and $F_{\mu\nu}$) are dynamic fields, operators acting on quantum states, not mere classical quantities.



\section{Gauge Invariance}

QED is a U(1) gauge theory. This means the Lagrangian \eqref{eq:qed_lagrangian} is invariant under local U(1) gauge transformations.

\subsection{Transformations}

\begin{align}

\psi(x) &\rightarrow \psi'(x) = e^{iq\alpha(x)}\psi(x) \label{eq:psi_transform}\\

A_\mu(x) &\rightarrow A'_\mu(x) = A_\mu(x) - \partial_\mu \alpha(x) \label{eq:A_transform}

\end{align}

where $\alpha(x)$ is an arbitrary differentiable scalar function of spacetime. The Manim visualization showed $\psi \to e^{i\alpha(x)}\psi$, implying $q=1$ or $\alpha(x)$ absorbed $q$. If $D_\mu = \partial_\mu + iqA_\mu$, then for the covariant derivative term $D_\mu \psi$ to transform "covariantly" (i.e., $D'_\mu \psi' = e^{iq\alpha(x)}D_\mu \psi$):

\begin{align*}

D'_\mu \psi' &= (\partial_\mu + iqA'_\mu) (e^{iq\alpha(x)}\psi(x)) \\

&= (\partial_\mu + iq(A_\mu - \partial_\mu\alpha(x))) e^{iq\alpha(x)}\psi(x) \\

&= e^{iq\alpha(x)} (iq(\partial_\mu\alpha)\psi + \partial_\mu\psi + iqA_\mu\psi - iq(\partial_\mu\alpha)\psi) \\

&= e^{iq\alpha(x)} (\partial_\mu + iqA_\mu)\psi = e^{iq\alpha(x)} D_\mu\psi

\end{align*}

This ensures that terms like $\bar{\psi}i\gamma^\mu D_\mu \psi$ are gauge invariant:

\begin{equation}

\bar{\psi}'(i\gamma^\mu D'_\mu)\psi' = (e^{-iq\alpha(x)}\bar{\psi}) (i\gamma^\mu e^{iq\alpha(x)}D_\mu\psi) = \bar{\psi}(i\gamma^\mu D_\mu)\psi

\end{equation}

The term $F_{\mu\nu}F^{\mu\nu}$ is also gauge invariant because $F_{\mu\nu}$ itself is invariant under \eqref{eq:A_transform}:

\begin{equation}

F'_{\mu\nu} = \partial_\mu A'_\nu - \partial_\nu A'_\mu = \partial_\mu (A_\nu - \partial_\nu\alpha) - \partial_\nu (A_\mu - \partial_\mu\alpha) = F_{\mu\nu} - (\partial_\mu\partial_\nu\alpha - \partial_\nu\partial_\mu\alpha) = F_{\mu\nu}

\end{equation}

since partial derivatives commute.



\subsection{Significance of Gauge Invariance}

\begin{itemize}

    \item \textbf{Charge Conservation:} Noether's theorem, applied to the global U(1) symmetry (where $\alpha(x)$ is a constant), leads to the conservation of electric charge, $\partial_\mu J^\mu = 0$.

    \item \textbf{Massless Photon:} Gauge invariance for the $A_\mu$ field dictates that a mass term for the photon ($m^2 A_\mu A^\mu$) would break gauge invariance, thus forcing the photon to be massless.

    \item \textbf{Interactions:} The requirement of local gauge invariance (where $\alpha(x)$ is a function of spacetime) naturally introduces the interaction term by promoting the partial derivative $\partial_\mu$ to the covariant derivative $D_\mu$, thereby introducing the gauge field $A_\mu$.

\end{itemize}



\section{Feynman Diagrams and Perturbative QED}

Since exact solutions to QED are intractable, physicists rely on perturbation theory. Feynman diagrams are a pictorial representation of terms in the perturbative expansion of interaction amplitudes (S-matrix elements).



\subsection{Basic Vertex}

The fundamental interaction in QED is between a fermion, an antifermion, and a photon. This is represented by a vertex in a Feynman diagram, derived from the interaction term $-q\bar{\psi}\gamma^\mu\psi A_\mu$ in the Lagrangian.

The animation shows two electron lines ($e^-$) approaching, exchanging a photon ($\gamma$), and then receding. This depicts electron-electron scattering (Møller scattering at tree level, if the exchanged photon is virtual).

\begin{itemize}

    \item \textbf{Electron lines (fermions):} Solid lines with arrows indicating the flow of charge (or fermion number). An arrow pointing forward in time is a particle; backward is an antiparticle. Represented by fermion propagators.

    \item \textbf{Photon line (boson):} A wavy or curly line. Represents the propagation of the electromagnetic force carrier. Represented by the photon propagator.

\end{itemize}



\subsection{Coupling Constant $\alpha$}

The strength of the electromagnetic interaction is characterized by the fine-structure constant $\alpha$:

\begin{equation}

\alpha = \frac{e^2}{4\pi\epsilon_0 \hbar c} \approx \frac{1}{137.036}

\end{equation}

In natural units ($\hbar=c=1, \epsilon_0=1$ in some conventions, or $e^2/(4\pi)$ in others):

\begin{equation}

\alpha = \frac{e^2}{4\pi} \approx \frac{1}{137}

\end{equation}

Each vertex in a Feynman diagram contributes a factor of $e$ (or $\sqrt{4\pi\alpha}$) to the amplitude. The probability (cross-section) is proportional to the amplitude squared, so it involves powers of $\alpha$. Since $\alpha \ll 1$, the perturbative expansion converges rapidly for many QED processes.



\section{Running of the Coupling Constant and Renormalization}

Classically, the coupling constant $\alpha$ is a fixed value. However, in quantum field theory, virtual particle-antiparticle pairs (like $e^-e^+$) can pop in and out of the vacuum.

\subsection{Vacuum Polarization}

For QED, virtual electron-positron pairs around a "bare" charge can screen it, effectively reducing the observed charge at large distances (low energies). Conversely, at short distances (high energies), one probes closer to the bare charge, and the effective coupling strength appears to increase. This phenomenon is called vacuum polarization.

\subsection{Renormalization Group Flow}

The change in the coupling strength with energy scale ($Q$) is described by the Renormalization Group Equations (RGEs). For QED, the one-loop beta function is:

\begin{equation}

\beta(\alpha) = Q \frac{d\alpha}{dQ} = \frac{2\alpha^2}{3\pi} + \mathcal{O}(\alpha^3)

\end{equation}

Since $\beta(\alpha) > 0$ for QED, the coupling $\alpha(Q)$ increases with increasing energy scale $Q$.

The solution to the RGE at one-loop for $Q^2 \gg m_e^2$ is approximately:

\begin{equation}

\alpha(Q^2) \approx \frac{\alpha(\mu^2)}{1 - \frac{\alpha(\mu^2)}{3\pi} \ln(Q^2/\mu^2)}

\end{equation}

where $\alpha(\mu^2)$ is the coupling at a reference scale $\mu^2$. For QED, this increase is very slow. The visualization plotting $\alpha$ vs. Energy Scale qualitatively demonstrates this phenomenon.



\subsection{Landau Pole}

The increasing nature of $\alpha(Q)$ in QED suggests that at some extremely high energy (the "Landau pole"), the coupling would become infinite, signaling a breakdown of perturbative QED. However, this energy is far beyond experimentally accessible scales and likely indicates that QED is an effective field theory, part of a more complete theory at very high energies (e.g., Grand Unified Theories or String Theory).



\section*{Conclusion}

The QED framework, from the relativistic metric and Maxwell's equations to its sophisticated Lagrangian formulation, gauge invariance, and renormalization phenomena, represents a triumph of 20th-century physics. It unifies special relativity and quantum mechanics to describe the interaction of light and matter with unprecedented precision, a story beautifully initiated by the mathematical constructs illustrated.



\end{document}

